\chapter{Armazém de dados}
\label{chap:data_warehouse}

\section*{}

Neste capítulo apresentam-se os armazéns de dados.

\section{Definição}
\label{section:definition}

Apesar de actualmente a definição precisa do termo armazenamento de dados ainda
ser tema de debate, a grande maioria dos investigadores e fornecedores de soluções
parece concordar que o mesmo poderá ser definido como um conjunto de conceitos,
metodologias, tecnologias e ferramentas utilizados com o propósito de disponibilizar
dados para aplicações orientadas ao suporte à decisão.


Nos ínicios da década de 90, Bill Inmon \citet{Inmon92} define o termo armazém
de dados como sendo uma colecção de dados orientada a um determinado assunto,
integrada, variável com o tempo e não volátil, usada como suporte no processo
da tomada de decisão. Esta, aos dias de hoje, parece ainda ser a definição que
mais consenso reune.

Tal como caracterizado em \citet{Inmon92}, um armazém de dados é:
\begin{itemize}
    \item Orientado ao assunto, isto é, ao contrário das bases de dados
    operacionais clássicas que se encontram organizadas em torno das operações
    diárias de uma companhia, a estrutura de um armazém de dados é baseada nos
    temas chave da organização em causa, e.g., vendas, clientes, pacientes,
    estudantes, permitindo assim um foco mais concreto sobre o dominio através
    da exclusão de informação que não seja relevante para o processo de tomada
    de decisão.

    \item Integrado, visto que, se propõem a reunir informação proveniente de
    diferentes sistemas externos e operacionais (fontes de dados). A integração
    em si é um processo complexo que normalmente implica um mapeamento de dados
    dissimilares com vista a obter uma base única e comum como fonte de verdade.
    Como exemplo, o sexo de um determinado indivíduo, poderá ser representado por
    'M' e 'F' ou '0' e '1' ou 'masc' e 'fem' quando na presença de fontes de
    informação heterogéneas. Por definição, os dados num armazém de dados
    deverão ser sempre mantidos de uma forma consistente e coerente.

    \item Variável com o tempo, que indica a possibilidade de representar
    diferentes valores do mesmo objecto, de acordo com as alterações que vai
    sofrendo ao longo do tempo, e.g., a média mensal de vendas de um determinado
    produto durante o período de dois anos. Ao invés da maioria das bases de dados
    de sistemas operacionais, dados históricos são de extrema relevância no
    contexto dos armazéns de dados.

    \item Não volátil, o último conceito essencial para se definir um armazém
    de dados, representa a não possibilidade de alteração ou remoção da
    informação uma vez inserida no mesmo, ou seja, o armazém de dados deverá,
    numa base periódica, ser actualizado com nova informação. Este aspecto é
    essencial no que diz respeito à capacidade de representação de informação
    num qualquer instante de tempo.
\end{itemize}

Por outro lado, de acordo com Ralph Kimball \citet{Kimball02}, um armazém de dados
poderá ser descrito como um aglomerado de todos os \textit{data marts} existentes
no contexto de uma empresa, que por sua vez, têm o seu foco direccionado para
os objectivos respeitantes ao negócio de cada departamento dentro da organização
em causa. Em \citet{Kimball96} o autor descreve um \textit{data mart} como sendo
um subconjunto do armazém de dados. O armazém de dados corresponde ao somatório
de todos os \textit{data marts}, sendo que cada um representa um negócio especifico
através de um modelo em estrela ou uma familia dos mesmos mas com diferentes
granularidades.
