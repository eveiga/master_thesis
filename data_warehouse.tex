\chapter{Armazém de dados}
\label{chap:data_warehouse}

\section*{}

Neste capítulo apresentam-se os armazéns de dados.

\section{Definição}
\label{section:definition}

Apesar de actualmente a definição precisa do termo armazenamento de dados ainda
ser tema de debate, a grande maioria dos investigadores e fornecedores de soluções
parece concordar que o mesmo poderá ser definido como um conjunto de conceitos,
metodologias, tecnologias e ferramentas utilizados com o propósito de disponibilizar
dados para aplicações orientadas ao suporte à decisão. Nos ínicios da década de
90, Bill Inmon \citet{Inmon92} define o termo armazém de dados como sendo uma
colecção de dados orientada a um determinado assunto, integrada, variável com o
tempo e não volátil, usada como suporte no processo da tomada de decisão. Esta,
aos dias de hoje, parece ainda ser a definição que mais consenso reune.

Tal como caracterizado em \citet{Inmon92}, um armazém de dados é:
\begin{itemize}
    \item Orientado ao assunto, isto é, ao contrário das bases de dados
    operacionais clássicas que se encontram organizadas em torno das operações
    diárias de uma companhia, a estrutura de um armazém de dados é baseada nos
    temas chave da organização em causa tais como vendas, clientes, pacientes,
    estudantes, permitindo assim um foco mais concreto sobre o dominio através
    da exclusão de informação que não seja relevante para o processo de tomada
    de decisão.
    \item Integrado
\end{itemize}

